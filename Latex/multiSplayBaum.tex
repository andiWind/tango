\documentclass[a4paper,12pt]{article}
\usepackage[utf8]{inputenc}
\usepackage[ngerman,german]{babel}
\usepackage[T1]{fontenc}
\usepackage{courier}
\usepackage{float}
\usepackage{amsmath,amsthm}
\usepackage{amssymb}
\usepackage{mathtools}
\usepackage{tabularx}
\usepackage{graphicx}
\usepackage{cite}
\usepackage{csquotes}
\usepackage[font=small,labelfont=bf]{caption}

%\usepackage[pages=some]{background} % Draft Wasserzeichen mit Option pages=all sonst pages=some



\title{Bachelorarbeit}
\author{
	Andreas Windorfer\\
}
\date{\today}


\begin {document}


\maketitle
\newpage

\newpage
\tableofcontents


\newpage


\section{Multisplay Baum}
Der Multisplay Baum ist eine Variation zum Tango Baum. Ein preferred path wird hier durch einen Splaybaum dargestellt. Amortisiert betrachtet, ist er $\log\left(\log\left(n\right)\right)$-competitive und garantiert $O\left(\log \left(n\right)\right)$ im worst case, bei einer einzelnen\\ {\textit{access}} Operation. $n$ steht wieder für der Anzahl der Knoten von $T$. Da der Splaybaum kein balancierter Baum ist, gibt es zusätzliche mögliche Zustände im Vergleich zu einem Tango Baum mit der gleichen Knotenzahl.
\begin{figure}[h]
	\centering
	\includegraphics[width= 0.8\textwidth]{"Medien/Multisplay/referenzTree"}
	\caption {Refernzebaum mit grün gezeichneten preferred paths }
	\label{fig:referenzTree}
\end{figure} 
\begin{figure}[h]
	\centering
	\includegraphics[width= 1\textwidth]{"Medien/Multisplay/pfadRepresentation"}
	\caption {Refernzebaum mit grün gezeichneten preferred paths }
	\label{fig:referenzTree}
\end{figure} 
 \subsection{Die \textit{access} Operation beim Multisplay Baum}





\newpage
\bibliography{literaturverzeichnis}
\bibliographystyle{unsrt}

\end {document}

