\documentclass[a4paper,12pt]{article}
\usepackage[utf8]{inputenc}
\usepackage[ngerman,german]{babel}
\usepackage[T1]{fontenc}
\usepackage{courier}
\usepackage{float}
\usepackage{amsmath,amsthm}
\usepackage{amssymb}
\usepackage{mathtools}
\usepackage{tabularx}
\usepackage{graphicx}
\usepackage{cite}
\usepackage{csquotes}
\usepackage[font=small,labelfont=bf]{caption}

%\usepackage[pages=some]{background} % Draft Wasserzeichen mit Option pages=all sonst pages=some



\title{Bachelorarbeit}
\author{

	Andreas Windorfer\\
}
\date{\today}


\begin {document}


\maketitle
\newpage

\tableofcontents
\newpage

\section{Dynamische Optimalität}

\subsection{BST Zugriffsalgorithmus }
Sei $T$ ein BST mit der festen Schlüsselmenge $K$. \textit{insert} und \textit{delete} bleiben also außen vor. Für \textit{access} werden als Parameter nur Schlüssel zugelassen die in $K$ enthaltenen sind. Für die Umsetzung der \textit{access(key $k$)} Operation  werden nur Algorithmen betrachtet, die sich wie folgt verhalten. 
\begin{enumerate} 
	\item Der Algorithmus verfügt über genau einen Zeiger $p$ in den BST. Dieser zeigt zu Beginn der Operation auf die Wurzel.  
	\item Der Algorithmus führt eine Folge der folgenden vier Operationen aus:
	\begin{itemize}
		\item Setze $p$ auf das linke Kind von $p$.
		\item Setze $p$ auf das rechte Kind von $p$.
		\item Setze $p$ auf den Vater von $p$.
		\item Führe eine Rotation auf $p$ aus.
	\end{itemize}  
	\item Nur wenn $p$ auf den Knoten $v$ mit Schlüssel $k$ zeigt, kann der Algorithmus terminieren und eine Referenz auf $v$ zurückgeben. 
\end{enumerate}
Solche Algorithmen bezeichnet man als \textbf{BST access algorithm}.
Jede der vier Operationen aus dem zweiten Punkt kann in konstanter Zeit durchgeführt werden. Die Kosten der teuersten Operation werden als Einheitskosten $u$ für alle vier Operationen verwendet. Für Punkt eins und drei werden ebenfalls jeweils diese Kosten veranschlagt. Sei $l$ die Länge der Folge aus Punkt zwei. Die Gesamtkosten für eine \textit{access} Operation sind damit $u \left(l + 2 \right) = O\left( l\right)$. 
  




\newpage
\bibliography{literaturverzeichnis}
\bibliographystyle{unsrt}

\end {document}


