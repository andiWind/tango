\documentclass[a4paper,12pt]{article}
\usepackage[utf8]{inputenc}
\usepackage[ngerman,german]{babel}
\usepackage[T1]{fontenc}
\usepackage{courier}
\usepackage{float}
\usepackage{amsmath,amsthm}
\usepackage{amssymb}
\usepackage{mathtools}
\usepackage{tabularx}
\usepackage{graphicx}
\usepackage{cite}
\usepackage{csquotes}
\usepackage[font=small,labelfont=bf]{caption}

%\usepackage[pages=some]{background} % Draft Wasserzeichen mit Option pages=all sonst pages=some



\title{Bachelorarbeit}
\author{
Andreas Windorfer\\
}
\date{\today}


\begin {document}


\maketitle
\newpage
Zusammenfassung
\newpage
\tableofcontents


\newpage


\section {Zipper-Baum}
Der Zipper-Baum basiert auf dem Tango-Baum und nutzt auch preferred paths aus einem lower-bound-tree $P$. Aufbau und Pflege der preferred paths in $P$ unterscheiden sich nicht vom Tango-Baum, wohl aber ihre Repräsentation im eigentlichen BST $T$. Der Zipper-Baum wurde in \cite{zipper} vorgestellt. Er ist ebenfalls $\log\left(\log\left(n\right)\right)$-competitive,  garantiert aber auch  $O\left(\log \left(n\right)\right)$ bei einer einzelnen \textit{access} Operation. $n$ steht wieder für de Anzahl der Knoten von $T$.  Das Verhalten der Operationen \textit{cut} und \textit{concatenate} unterscheidet sich deutlich. 


\subsection{Repräsentation eines preferred path}
Ein Hilfsbaum $H$ zur Repräsentation eines preffered path $P_p = p_1,p_2,..,p_p$ wird in einen \textbf{zipper} und einem \textbf{bottom tree} unterteilt. Enthält der preferred path nicht mehr als $O\log\left(\log\left(n\right)\right)$ Knoten, besteht der Hilfsbaum allein aus dem zipper. Die Anzahl der Knoten des zipper liegt in \\$\left[\log\left(\log\left(n\right)\right) / 2, 2 \log\left(\log\left(n\right)\right)  \right] $, wenn ein bottom tree existiert, ansonsten in $\left[0, 2 \log\left(\log\left(n\right)\right)  \right] $. Der bottom tree ist ein balancierter BST, der genau die Schlüssel aus $P_p$ enthält, die im zipper fehlen. Enthält der zipper $q$ Knoten, dann entsprechen deren Schlüsseln denen aus dem Pfad $p_1, p_2,..,p_q$. Der zipper ist ein BST und die Wurzel des bottom tree ist das Kind eines Knotens aus dem zipper, so dass $H$ ein BST ist. \\
 Die Konstruktion des zipper ist so ausgelegt, dass innerhalb konstanter Zeit von der Wurzel von $H$ auf die Wurzel des bottom tree zugegriffen werden kann. Es gibt in der Regel mehrere mögliche Darstellungen eines zipper zu $P_p$. Ein zipper $z$ besteht ebenfalls wieder aus zwei Bestandteilen dem \textbf{oberen zipper} $z_1$  und dem \textbf{unterem zipper} $z_2$. Insgesamt müssen in $z$ zumindest  $\log\left(\log\left(n\right)\right) / 2$ enthalten sein. Pro Bestandteil dürfen maximal $\log\left(\log\left(n\right)\right)$ Knoten enthalten sein. Sei $a_1$ die Anzahl der Knoten in $z_1$ und $a_2$ die in $z_2$, so dass die genannten Anforderungen eingehalten werden. Konstellationen in denen das nicht möglich ist, bleiben zunächst außen vor. In $z_1$ sind die Schlüssel der Knoten des Pfades $P_1 = p_1, p_2,.., p_{a_1}$ enthalten. In $z_2$ die aus $P_2 = p_{a_1 + 1}, p_{a_1 + 2},..,p_{a_1 + a_2} $. \\
 $P_1$ wird in \textbf{zig Segmente} und \textbf{zag Segmente} unterteilt. zig Segmente entsprechen den längst möglichen Teilpfaden von Knoten mit linken Kindern in $P_p$. zag Segmente entsprechen den längst möglichen Teilpfaden von Knoten mit rechten Kindern in $P_p$. Enthält $P_2$ das Blatt aus $P_p$, wird dieses dem Segment seines Vaters zugeordnet. In Abbildung  \ref{fig:preferredPathZigZag} sind zig und zag Segmente dargestellt.
  Sei $S_1$ die Folge der Knoten der zig Segmente, aufsteigend sortiert nach der Tiefe und $S_2$ die Folge der Knoten der zag Segmente, aufsteigend sortiert nach der Tiefe der enthaltenen Knoten. Ist $u$ der tiefste Knoten eines zig bzw. zag Segmentes, so können in den Segmenten mit Knoten größerer Tiefe nur noch größere bzw. kleinere Schlüssel enthalten sein, vergleiche Abschnitt \ref{dd} und Abbildung \ref{fig:preferredPathZigZag}. Deshalb müssen die Knoten in $S_1 \circ S_2$ aufsteigend sortiert nach Schlüssel sein. Da die Knoten in $S_1$ bzw. $S_2$ aber auch aufsteigend bzw. absteigend nach Tiefe sortiert sind, können aus $S_1 \circ S_2$ Intervalle für die Schlüsselmengen, von Pfaden in $P_p$ mit Endknoten $p_p$ abgeleitet werden, siehe Abbildung \ref{fig:preferredPathZigZag}, diese werden in den nachfolgend vorgestellten Operationen benötigt.\\
  Sei nun $l$ der Knoten in $S_1$ mit der größten Tiefe, und $r$ der Knoten in $S_2$ mit der größten Tiefe. Der Knoten mit Schlüssel $\mathit{key}\left(l\right)$  ist die Wurzel von $H$. Der linke Teilbaum von $l$ enthält die Schlüssel der Knoten in $S_1$ mit einer Tiefe $\leq a_1$. 
 
 
 
 
 
  
 
\begin{figure}[h]
	\centering
	\includegraphics[height= 0.7\textwidth]{"Medien/Zipper/preferredPathZigZag"}
	\caption{zig Segmente sind grün dargestellt. zag Segmente blau }
	\label{fig:preferredPathZigZag}
\end{figure}
\newpage
\bibliography{literaturverzeichnis}
\bibliographystyle{unsrt}

\end {document}