\documentclass[a4paper,12pt]{article}
\usepackage[utf8]{inputenc}
\usepackage[ngerman,german]{babel}
\usepackage[T1]{fontenc}
\usepackage{courier}
\usepackage{float}
\usepackage{amsmath,amsthm}
\usepackage{amssymb}
\usepackage{mathtools}
\usepackage{tabularx}
\usepackage{graphicx}
\usepackage{cite}
\usepackage{csquotes}
\usepackage[font=small,labelfont=bf]{caption}

%\usepackage[pages=some]{background} % Draft Wasserzeichen mit Option pages=all sonst pages=some



\title{Bachelorarbeit}
\author{
Andreas Windorfer\\
}
\date{\today}

\begin {document}



\maketitle
\newpage

\tableofcontents
\newpage

\section{Tango Baum}
Der Tango Baum ist ein aus BSTs, den \textbf{auxiliary trees}, zusammengesetzter BST. Auf die Anforderungen an die auxiliary trees wird in \ref{aufbauDesTango} eingegangen und mit dem Rot-Schwarz-Baum wird eine mögliche Variante noch detailliert vorgestellt. Der Tango Baum wurde in \cite{demainDinamicOpti} beschrieben, inklusive eines Beweises über seine $\mathit{lg} \, \mathit{lg} \, n $-competitiveness. Ebenfalls in \cite{demainDinamicOpti} enthalten ist eine Variation der Interleave Bound von Wilber. Da diese für das Verständnis des Tango Baumes wesentlich ist, wird mit ihr gestartet, bevor es zur Beschreibung der Struktur selbst kommt. 


\subsection{Eine Variation der Interleave Lower Bound}
\subsection{Aufbau des Tango Baum} \label{aufbauDesTango}
\subsection{Die \textit{access} Operation beim Tango Baum}
\subsection{Laufzeitanalyse}
\newpage
\bibliography{literaturverzeichnis}
\bibliographystyle{unsrt}

\end {document}


