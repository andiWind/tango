\documentclass[11pt, a4paper]{article}

\usepackage[utf8]{inputenc}
\usepackage[T1]{fontenc}
\usepackage{lmodern}
\usepackage[ngerman]{babel}
\usepackage{babelbib}
\usepackage{csquotes}
\usepackage[draft]{hyperref}    % korrektes Management von
                                % PDF-Einstellungen

\usepackage{amsmath,amssymb,amsthm}

\newtheorem{definition}{Definition}
\newtheorem{satz}{Satz}

% Dieser Block wird für die *Präsentation* der Lösung benötigt,
% nicht für die Lösung selbst. Falls dieser Teil dringelassen
% wird, so muss (PDF)LaTeX via
%
%     pdflatex -shell-escape ml3.tex
%
% aufgerufen werden, da das "minted" Paket diese Funktionalität
% benötigt.
%
\usepackage{minted}

\begin{document}

%%%%%%%%
%
% Der folgende Teil gehört nicht zur eigentlichen Lösung.
%
%%%%%%%%

\setcounter{page}{0}
\section*{Lösungsvorschlag zur Einsendeaufgabe 3, Kurs 1602}

Dies ist der Lösungs\emph{vorschlag} zur Einsendeaufgabe \#3,
Kurs 1602. Einsendeaufgabe \#3 ist personalisiert, die hier
präsentierte Lösung weicht deshalb wahrscheinlich von Ihrer
Aufgabenstellung geringfügig ab, etwa durch andere Formeln in
Abschnitt~\ref{sec:mathe}, eine andere Tabelle~\ref{tab:2} oder
andere Einträge im Literaturverzeichnis. Ihre Aufgabenstellung
sollte aber mit der hier präsentierten Aufgabenstellung
vergleichbar sein.

\pagebreak

%%%%%%%%
%
% Ab hier gehört wieder alles zur eigentlichen Lösung.
%
%%%%%%%%

\section{Vorlage 999 zur Einsendeaufgabe 3, Kurs 1602}
\label{sec:1}

\subsection{Die Aufgabe}\label{sec:11}

In dieser Kurseinheit wird LaTeX vorgestellt. LaTeX ist ein
\emph{Textsatzprogramm}, welches das Erstellen von
wissenschaftlichen Texten erleichtert. In dieser Aufgabe sollen
Sie das vorliegende Dokument möglichst genau mit LaTeX
nachbilden. So soll insbesondere die Aufteilung in Kapitel
(hier also in Kapitel \ref{sec:1}, Kapitel~\ref{sec:11}, usw.),
die Formatierungen und der Text \textbf{exakt} nachgebildet
werden. Verwenden Sie die Dokumentenklasse \enquote{article}
mit der Option \enquote{a4paper}.

\subsection{Die Lösung}

Bitte senden Sie folgende Dokumente als Teil der Lösung ein:
\begin{enumerate}
	\item die \textbf{original PDF Vorlage} (also diese
	Datei, als Lösung zu \enquote{Teilaufgabe 3a}),
	
	\item das von Ihnen erstellte \textbf{LaTeX Dokument}
	(also die .tex Datei, als Lösung zu \enquote{Teilaufgabe
	3b}),
	
	\item die von Ihnen erstellte \textbf{Literatur
	Datenbank} (also die .bib Datei, als Lösung zu
	\enquote{Teilaufgabe 3c}),
	
	\item das von Ihnen \textbf{kompilierte Dokument} (also
	die erzeugte .pdf Datei, als Lösung zu
	\enquote{Teilaufgabe 3d}).
\end{enumerate}

\section{Tabellen}

Tabelle~\ref{tab:2} gibt eine kleine Übersicht über verschiedene
Charaktere und Gruppierungen in \emph{Der Herr der Ringe}.

\begin{table}[b]
	\centering
	\begin{tabular}{|r|c|c|c|c|}
		\hline 	 	 	
		& Gollum & Sauron & Legolas & Gandalf\\
		\hline\hline
		Hobbit & ja & nein & nein & nein\\
		\hline
		Ringträger & am Finger  & am Finger  & nein & in
		der Hand\\
		\hline
		Gemeinschaft des Ringes & nein & nein & ja & ja\\
		\hline
	\end{tabular}
	\caption{Verschiedene Charaktere und Gruppierungen in
	\emph{Der Herr der Ringe}.}
	\label{tab:2}
\end{table}

\section{Formeln}
\label{sec:mathe}   % wird für die Anmerkung zum
		    % Lösungsvorschlag benötigt, nicht für die
		    % Lösung selbst

Die bekannten Fibonacci-Zahlen $F_n$ sind rekursiv definiert
durch $F_1=F_2=1$ und $F_n=F_{n-1}+F_{n-2}$ für $n\geq 3$.
Fibonacci-Zahlen treten sowohl in der Natur als auch in vielen
theoretischen Anwendungen auf. Die Formel in Satz~\ref{satz:3}
wurde von verschiedenen Mathematikern im 18. und 19. Jahrhundert
entdeckt.

\begin{satz}\label{satz:3}
 Es sei $\phi=\frac{1+\sqrt{5}}{2}$ (der goldene Schnitt) und
 $\psi=\frac{1-\sqrt{5}}{2}$. Dann gilt
 \[\forall n\in\mathbb{N}:\quad F_n = \frac{\phi^n-\psi^n}%
   {\sqrt{5}}.\]
\end{satz}

\section{Zitieren}

In Ihrer Abschlussarbeit werden Sie existierende Literatur
bearbeiten und zitieren müssen. Das korrekte Formatieren
übernimmt LaTeX (beziehungsweise BibTeX) für Sie. Dabei werden
Sie vermutlich nicht aus dem Buch von Tolkien~\cite{Tolkien54}
zitieren, aber möglicherweise aus einem Buch zum
Übersetzerbau~\cite{ALSU06}.

\bibliographystyle{babplain-fl}
\bibliography{ml3}

%%%%%%%%
%
% Der folgende Teil gehört nicht mehr zur Lösung; er dient der
% Anzeige der LaTeX- und BibTeX-Quelldatei in der PDF-Ausgabe.
%
%%%%%%%%

\pagebreak

\section*{LaTeX-Code: \texttt{ml3.tex}}
\inputminted[breaklines, encoding=utf8, frame=single]%
            {LaTeX}{ml3.tex}

\section*{BibTeX-Code: \texttt{literatur.bib}}
\inputminted[breaklines, encoding=utf8, frame=single]%
            {BibTeX}{ml3.bib}

\end{document}
