\documentclass[a4paper]{article}
\usepackage{amsmath}
\usepackage{amsthm}
\usepackage[utf8]{inputenc}
\usepackage[T1]{fontenc}
\usepackage{lmodern}
\usepackage[ngerman]{babel}
\usepackage{amssymb}
\usepackage[justification=raggedright,
singlelinecheck=false]{caption}
\usepackage{babelbib}
 \usepackage[autostyle=true,german=quotes]{csquotes}
\begin{document}

	\section{Vorlage 113 zur Einsendeaufgabe 3, Kurs 1602}

	\subsection{Die Aufgabe}

	\noindent In dieser Kurseinheit wird LaTeX vorgestellt. LaTeX ist ein \textit{Textsatzprogramm}, welches das Erstellen von wissenschaftlichen Texten erleichtert. In dieser Aufgabe sollen Sie das vorliegende Dokument möglichst genau mit LaTeX nachbilden. So soll insbesondere die Aufteilung in Kapitel (hier also in Kapitel 1, Kapitel 1.1, usw.), die Formatierungen und der Text \textbf{exakt}
    nachgebildet werden. Verwenden Sie die Dokumentenklasse \enquote{article} mit der Option \enquote{a4paper}.
	
	\subsection{Die Lösung}
   \smallbreak
	Bitte senden Sie folgende Dokumente als Teil der Lösung ein:
	
	\begin{enumerate}
	 	\item 
	 	die \textbf{original PDF Vorlage} (also diese Datei, als Lösung zu „Teilaufgabe 3a“),
	 	
	 	\item 
	 	das von Ihnen erstellte \textbf{LaTeX Dokument}(also die .tex Datei, als Lösung  zu „Teilaufgabe 3b“),
	 
	 	\item 
	 	die von Ihnen erstellte \textbf{Literatur Datenbank}(also die .bib Datei, als Lösung zu „Teilaufgabe 3c“),
	 	
	 	\item 
		das von Ihnen \textbf{kompilierte Dokument}(also die erzeugte .pdf Datei, als Lösung zu „Teilaufgabe 3d“).
    \end{enumerate}	

	\section{Tabellen}
	
	Tabelle 1 gibt eine kleine Übersicht über verschiedene Charaktere und Gruppierungen in \textit{Der Herr der Ringe}.
	
		
	\section{Formeln}
	
	Die bekannten Fibonacci-Zahlen $F_n$ sind wie folgt definiert.\medskip
	
	
	\noindent \textbf{Definition 1.} Es sei $F_1=F_2=1 \text{\textit{ und }} F_n=F_{n-1}+F_{n-2} \text{\textit{ für }} n\geq3$.\bigskip

	\begin{table}[b]
		\centering
		\begin{tabular}{|c|c|c|c|}
			\hline
			& Hobbit & Ringträger & Gemeinschaft des Ringes \\
			\hline
			\hline
			Gollum & ja & am Finger & nein \\
			\hline
			Sauron & nein & am Finger & nein \\
			\hline
			Legolas & nein & nein & ja \\
			\hline
			Gandalf & nein & in der Hand & ja \\
			\hline	
		\end{tabular}	
		\caption{Verschiedene Charaktere und Gruppierungen in \textit{Der Herr der Ringe}.}
	\end{table}
	
	Fibonacci-Zahlen treten sowohl in der Natur als auch in vielen theoretischen Anwendungen auf. Die folgende Formel für die in Definition 1 wurde von verschiedenen Mathematikern im 18. und 19. Jahrhundert entdeckt. Es sei $\phi=\frac{1+\sqrt{5}}{2}$ (der goldene Schnitt) und $\psi=\frac{1-\sqrt{5}}{2}$ . Dann gilt
	\[\forall{n}\in\mathbb{N}:\quad F_n=\frac{\phi^n-\phi^n}{\sqrt{5}}\text{.}\] 
	
	
	

	
	\section{Zitieren}
    \bibliographystyle{babplain-fl}
	In Ihrer Abschlussarbeit werden Sie existierende Literatur bearbeiten und zitieren müssen. Das korrekte Formatieren übernimmt LaTeX (beziehungsweise BibTeX) für Sie. In der Komplexitätstheorie zitieren Sie möglicherweise aus einer wichtigen Arbeit von Karp ~\cite{2}  oder aus dem Buch von ~\cite{1}.
	
    
    \bibliography{lit}
    
	
\end{document}